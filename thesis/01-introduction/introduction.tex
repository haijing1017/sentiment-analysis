	For start let us define what is sentiment analysis and why could we potentially depend on it. Sentiment analysis is the process of computationally identifying and categorizing opinions expressed in a piece of text, especially in order to determine whether the writer's attitude towards a particular topic, product, etc. is positive, negative, or neutral [Oxford dictionary definition]. How could we in future depend on it? With growth of applying marketing through social media gives us opportunity to seize information about products spread around the media. By analyzing customer's input about a certain campaign, product or brand's strategy a company could predict future trends, decrease costs and increase profit.
	
	Within the described context, this project aims to give statistical comparison of open source APIs used to determine sentiment on small dataset consisting of Facebook posts and related comments. The main focus will be on showing results of the analysis, as well as which is the most efficient sentiment analysis API.

\section{Structure}
\blindtext

\begin{itemize}
	\item 
	In the chapter \ref{ch:state-of-the-art}  we will present the reasons for doing sentiment analysis within a company or a non-profit organization as well as today's successful solutions and their consequences. We will mention about most used commercial solutions and their strengths and weaknesses. Afterwards we will list some of most known open source libraries.
	\item 
	In the chapter \ref{ch:sentiment-analysis-workflow} is dedicated to describing the workflow of our project. Describing all the steps of the process from collecting the data, analyzing it with different APIs, determining real sentiment, and finally calculating the APIs accuracy.
	\item 
	In the chapter \ref{ch:framework} we will give an overview of the Framework we have built with all its components. Describing the use of Django REST Framework, libraries we have included, etc. NOT FINISHED 
	\item 
	In the chapter \ref{ch:results} is the part where the project results are supported with interpretations. We will present all the statistical results we have obtained after comparing different APIs
	\item 
	In the chapter \ref{ch:conclusion} will be finalize the purpose of the project and conclude our findings.
	\item 
	In the chapter \ref{ch:future-work} we will mention about future improvements such as doing clustering in sentiment analysis, finding a better spam detection solution and eventually training a model that would accommodate our domain problem.

\end{itemize}