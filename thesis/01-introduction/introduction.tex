	Sentiment analysis is the process of computationally identifying and categorizing opinions expressed in a piece of text, especially in order to determine whether the writer's attitude towards a particular topic, product, etc. is positive, negative, or neutral [Oxford dictionary definition]. How could we in future depend on it? With growth of applying marketing through social media gives us opportunity to seize information about products spread around the media. By analyzing customer's input about a certain campaign, product or brand's strategy a company could predict future trends, decrease costs and increase profit.
	
	Within the described context, this project aims to give statistical comparison of open source APIs used to determine sentiment on small dataset consisting of social media posts and related comments in the field of fashion. Given the increasing number of Sentiment Analysis APIs available online, our framework is meant to help developers of new applications, which need to leverage on sentiment services, to compare different resources and choose the one(s) that best fits the application requirements. 
	Four different API endpoints were chosen to be compared on the grounds of them being free, openly available and easily programmatically accessible.
	However, given the frameworks extensibility, it can be a useful mean for the comparison in general of sentiment analysis resources -- not only the ones that we exploited in the experimental part of our work. 

\section{Structure}

\begin{itemize}
	\item 
	In  chapter \ref{ch:state-of-the-art}  we will present the reasons for doing sentiment analysis within a company or a non-profit organization, today's successful solutions and their consequences. We will mention about some of the used commercial solutions, their strengths and weaknesses. Afterwards we will list some open source libraries.
	\item 
	Chapter \ref{ch:sentiment-analysis-workflow} is dedicated to describing the workflow of our project by illustrating all the steps of the process from collecting the data, analyzing it with different APIs, determining real sentiment, and finally calculating the APIs accuracy.
	\item 
	In chapter \ref{ch:framework} we will give an overview of the Framework we have built with all its components. We describe the high level architecture of the system and go down to explaining the role of each component. We will finish the chapter by giving a sketch of the user interface.
	\item 
	Chapter \ref{ch:results} is the part where the project results are supported with interpretations. We will present all the statistical results we have obtained after comparing different APIs.
	\item 
	Chapter \ref{ch:conclusion} will finalize the purpose of the project and conclude our findings. Also we will mention about future improvements such as doing clustering in sentiment analysis, finding a better spam detection solution and eventually training a model that would accommodate our domain problem.

\end{itemize}