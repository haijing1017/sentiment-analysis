This chapter describes state of the art of sentiment analysis in social media.
Chapter consists of three sections, each of them trying to bring closer the need of sentiment analysis in current market:

\begin{enumerate}
	\item Need for sentiment analysis
	\item Application of sentiment analysis in various companies and non-profit organizations
	\item Most used tools for sentiment analysis
\end{enumerate}

\section{Need for sentiment analysis}
With growth of people's interaction and company's advertisements through social media, we have come to the point of realizing that people sharing opinions could help us "predict" stock market and as well follow current trends by guiding the market according to the customers input.
Customers nowadays have endless ways to interact with brands which could help increasing brand's awareness but if not properly analyzed could also lead to obtaining not quite accurate view of customer's satisfaction.
The idea of analyzing customer opinion has driven companies to search for an automated way of understanding what message are customers sharing online. The main network of spreading opinions is social media. Almost every tweet, comment, re-share or  review gives an information that could guide a company towards better planning , optimizing production and better stock managing.
Reason for finding an automated way of analyzing customer's opinion comes from a problem of big data being generated each day which makes impractical of doing human analysis of each user input. Leaving the big data problem aside, brings us to another issue; being able to beat natural language processing challenge. Reason for making the task harder is that user input might be informal, "slang like content with emojis , hash tags, even full with sarcastic sentences which would lead to unreliable results of sentiment analysis.

\section{Application of sentiment analysis in various companies and non-profit organizations}

\section{Most used tools for sentiment analysis}

Commercial solutions:\\

Google Analytics
Radian6
Brandwatch\\

Open source solutions:\\

NLTK 
Stanford's CoreNLP
Text-Processing\\
