The most prevalent web development paradigm in the recent years has to be designing software components as APIs. 
The term API is short for Application Programming Interface and is essentially an interface to retrieve and manipulate data. But unlike a graphical user interface (from now on GUI) whose end users are people, API's users are other applications. 
Having single point of entry which abstracts and manages resources in a consistent manner provides lot of benefits in terms of maintainability, automation and future development.

That is why we took the API-first approach when structuring our sentiment analysis framework. 
The idea was to build a RESTful API on top of the provided sample database and use it for multiple purposes, namely, a web app, API's GUI or any scripts and third party applications that might need to access database records. 
However, during one development iteration the requirements somewhat changed.
It became clear that the collection of scripts used for sentiment analysis (see Section \ref{sec:sentiment-prediction-workflow}) would provide more value if their data requests didn't go through an API but by making calls directly to the database. 
Even though this decision might seem to counter all previously listed benefits, it was important to realize that our efforts weren't an independent endeavor but a possible extension of our university's existing framework for managing social media accounts. 
Considering the redefined context in which we were developing, it made sense to design the collection of server-side scripts so that they depend only on the specifics of the provided database schema.
This way, should our attempts yield satisfactory results, the collection of scripts could easily be deployed and reused in the university's existing framework. 
Our goal then became twofold, first designing an API-independent script bundle and second designing an API-dependent web application and its GUI. 
The purpose of the two latter services was mainly to aid development and to have results presented in an intuitive manner.
Finally, a high level design of the system is shown in Figure \ref{fig:framework-design}.
\usetikzlibrary{shapes.geometric, arrows, fit}
\tikzstyle{rect-small} = [rectangle, thick, minimum width=3.2cm, minimum height=1cm, text centered, draw=black, align=center]
\tikzstyle{rect-large} = [rectangle, thick, minimum width=3.2cm, minimum height=1.5cm, text centered, draw=black, align=center]
\tikzstyle{db}         = [cylinder, thick, shape border rotate=90, draw,minimum height=1.5cm, minimum width=2cm, shape aspect=.25]
\tikzstyle{cloud-api}  = [cloud, thick, draw,cloud puffs=10,cloud puff arc=120, aspect=2.5, align=center] 

\tikzstyle{arrow}    = [thick,<->,>=stealth]
\tikzstyle{arrow-to} = [thick,->,>=stealth]

\begin{figure}[ht]
  \centering
\begin{tikzpicture}[node distance=2cm]
\node (database) [db]  	{Database};

\node (scripts)         [rect-large, above right of=database, xshift=2.2cm, yshift=0.5cm] {Sentiment\\Analysis Scripts};
\node (google-cloud)    [cloud-api, right of=scripts, xshift=3cm, yshift=-1.4cm]          {Google\\Translate API};
\node (sentiment-cloud) [cloud-api, above of=google-cloud, yshift=0.4cm]                  {Sentiment\\APIs};

\node (business) [rect-small, below of=scripts,  yshift=-2cm] {Business Logic};
\node (api)      [rect-small, below of=business, yshift=1cm]  {API};

\node (apigui)   [rect-large, below of=api]                                    {RESTful\\API GUI};
\node (web-app)  [rect-large, below left of=api, xshift=-2.2cm, yshift=-0.6cm] {Web\\Application};
\node (3rd)      [rect-large, below right of=api, xshift=2.2cm, yshift=-0.6cm] {Third Party\\Applications};

\node (web-server) [below of=3rd, xshift=1cm, yshift=0.5cm] { web server};
\node [draw=black!100, dashed, fit={(business) (api) (apigui) (web-app) (3rd) (web-server)}, yshift=0.2cm, minimum width=11.5cm, minimum height=6cm] {};


\draw [arrow]    (database) |- (business);
\draw [arrow]    (database) |- (scripts);
\draw            (business) -- (api);
\draw [arrow]    (3rd)      |- (api);
\draw [arrow]    (apigui)   -- (api);
\draw [arrow-to] (web-app)  |- (api);
\draw [arrow-to] (scripts)  -- (google-cloud);
\draw [arrow-to] (scripts)  -- (sentiment-cloud);

\end{tikzpicture}
  \caption{Sentiment analysis framework}
  \label{fig:framework-design}
\end{figure}

