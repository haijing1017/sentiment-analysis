\subsection{Sentiment analysis scripts\label{sec:sentiment-analysis-workflow}}

The logic of all workflows described in Chapter \ref{ch:sentiment-analysis-workflow} are implemented as a collection of python scripts.
If we were to take the top down approach as we did in Chapter \ref{ch:sentiment-analysis-workflow}, we can start identifying specific nodes in workflow flowcharts to single scripts.

Predicting sentiment, determining real sentiment and the prediction evaluation parts of Figure \ref{fig:analysis-workflow} are realized via 
\emph{\wrapunderscore{automated\_sentiment\_analysis.py}}, 
\emph{\wrapunderscore{update\_real\_sentiment\_and\_spam.py}} and 
\emph{\wrapunderscore{evaluate\_api\_performance.py}}, respectively.

% -- API EVALUATION --
\subsubsection*{API evaluation script}
\noindent The \emph{\wrapunderscore{evaluate\_api\_performance.py}}\ script can be evoked via the command line and it stores the results in \inlinecode{im\_sentiment\_api\_stats} database table. It accepts the following optional arguments:

\begin{description}[labelindent=0.7cm, leftmargin=1.7cm]
\singlespacing
\item[--help ] shows all available arguments and exits
\item[--api \textless name\textgreater] 
	specifies which API $\in$ \lcb 
	sentiment\_api1, 
	sentiment\_api1\_en, 
	sentiment\_api2, 
	sentiment\_api2\_en, 
	sentiment\_api3, 
	sentiment\_api4\rcb\ to evaluate. If not specified, script evaluates all APIs 
\item[--metric \textless name\textgreater ] specifies which
	specifies which metric $\in$ \lcb
	precision, accuracy, recall\rcb\ to calculate. If not specified, script calculates all metrics
\item[--spam] calculates performance metric(s) for specified API(s) taking into consideration all comments regardless weather or non they were tagged as spam
\item[--no-spam] calculates performance metric(s) for specified API(s) taking into consideration only the comments not tagged as spam. 
If neither \inlinecode{no-spam} nor \inlinecode{spam} arguments were specified, the metrics are calculated for both cases
\item[--emoji] calculates performance metric(s) for specified API(s) of sentiment statistics that accounted for emojis. 
\item[--no-emoji] calculates performance metric(s) for specified API(s) of sentiment statistics that didn't accounted for emojis. 
If neither \inlinecode{no-emoji} nor \inlinecode{emoji} arguments were specified, the metrics are calculated for both cases
\end{description}
\textbf{Usage and sample output}


\begin{verbatim}
$ python evaluate_api_performance.py --metric accuracy --emoji --spam
----------------------------------------------------
Calculations for api: sentiment_api1_emoji with spam
Real sentiment distribution {
  "positive": "43.51%", 
  "neutral": "45.61%", 
  "negative": "10.88%"
}
True positives: {'positive': 1205, 'neutral': 2167, 'negative': 177} 
False negatives: {'positive': 1439, 'neutral': 605, 'negative': 484} 
False positives: {'positive': 517, 'neutral': 1694, 'negative': 317} 
Total sentiment predictions: 6077 
Accuracy: 0.584000 
- - - - - - - - - - - - - - - - - - - - - - - - - -
Updating table im_sentiment_api_stats
Setting `accuracy_with_spam` = 0.584 
Where `api_id` = 'sentiment_api1_emoji' 
... 1 row affected
----------------------------------------------------
\end{verbatim}


% -- REAL SENTIMENT --
\subsubsection*{Script for determining real sentiment}
\noindent The \emph{\wrapunderscore{update\_real\_sentiment\_and\_spam.py}} script can be evoked via the command line and stores the results in \.
For each comment the script displays its content and prompts user for input to determine: 
 \begin{enumerate}
  \item whether or not the comment is only a user tag (e.g @Anna)
  \item whether or not the comment is spam and if so, to specify a type
  \item comment sentiment
\end{enumerate}
The script accepts the following optional arguments:
\begin{description}[labelindent=0.7cm, leftmargin=1.7cm]
\singlespacing
 \item[--help ] shows all available arguments and exits
 \item[-ideq \textless nbs\textgreater ] specifies ids of the comments for which \\ we want to determine the real sentiment
 \item[-idgt \textless nb\textgreater] gets all comments that satisfy $id > nb$
 \item[-idlt \textless nb\textgreater] gets all comments that satisfy $id < nb$
\end{description}
\textbf{Usage and sample output}

\begin{verbatim}
$  python update_real_sentiment_and_spam.py -ideq 6
--------------------------------------------------
Comment_id: 6
Content: Bellissime!
English translation: Beautiful!
- - - - - - - - - - - - - - - - - - - - - - - - - 
Is this comment ONLY a mention? (y/n): n
Updating table im_commento_sentiment
Setting `is_mention` = '0' 
Where idcommento = 6 
... 1 row affected
- - - - - - - - - - - - - - - - - - - - - - - - - 
Is this comment spam? (y/n): n
Updating table im_commento_sentiment
Setting `spam` = '{"type": "", "is_spam": false}' 
Where idcommento = 6 
... 1 row affected
- - - - - - - - - - - - - - - - - - - - - - - - - 
Determine the real_sentiment:
pos/neg/neu/mix? pos
Updating table im_commento_sentiment
Setting `real_sentiment` = '{
  "sentiment_stats": {
    "positive": 1, 
    "neutral": 0, 
    "negative": 0
  }, 
  "sentiment_label": "positive"
}'
Where idcommento = 6 
... 1 row affected
--------------------------------------------------
\end{verbatim}




