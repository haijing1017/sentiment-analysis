\subsection{Sentiment analysis scripts\label{sec:sentiment-analysis-workflow}}

The logic of all workflows described in Chapter \ref{ch:sentiment-analysis-workflow} is implemented as a collection of python scripts.
If we were to take the top down approach as we did in Chapter \ref{ch:sentiment-analysis-workflow}, we can start mapping specific nodes in workflow flowcharts to single scripts.

Predicting sentiment, determining real sentiment and performance evaluation parts of Figure \ref{fig:analysis-workflow} are realized via 
\emph{\wrapunderscore{automated\_sentiment\_analysis.py}}, 
\emph{\wrapunderscore{update\_real\_sentiment\_and\_spam.py}} and 
\emph{\wrapunderscore{evaluate\_api\_performance.py}}, respectively.

% -- API EVALUATION --
\subsubsection*{API evaluation script}
\noindent The \emph{\wrapunderscore{evaluate\_api\_performance.py}}\ script can be invoked via the command line and it stores the results in \inlinecode{im\_sentiment\_api\_stats} database table. It accepts the following optional arguments:

\begin{description}[labelindent=0.7cm, leftmargin=1.7cm]
\singlespacing
\item[--help -h ] shows all available arguments and exits
\item[--api \textless name\textgreater] 
	specifies which API $\in$ \lcb 
	sentiment\_api1, 
	sentiment\_api1\_en, 
	sentiment\_api2, 
	sentiment\_api2\_en, 
	sentiment\_api3, 
	sentiment\_api4\rcb\ to evaluate. If not specified, script evaluates all APIs 
\item[--metric \textless name\textgreater ]
	specifies which metric $\in$ \lcb
	precision, accuracy, recall\rcb\ to calculate. If not specified, all are calculated
\item[--spam] 
	calculates performance metric(s) for specified API(s) taking into account all comments regardless weather or non they were tagged as spam
\item[--no-spam] 
	calculates performance metric(s) for specified API(s) taking into account only the comments not tagged as spam. 
	If neither \inlinecode{no-spam} nor \inlinecode{spam} arguments were specified, the metrics are calculated for both cases
\item[--emoji] 
	calculates performance metric(s) for specified API(s) of sentiment statistics from \inlinecode{im\_commento\_sentiment\_emoji} table
\item[--no-emoji] 
	calculates performance metric(s) for specified API(s) of sentiment statistics from \inlinecode{im\_commento\_sentiment} table
	If neither \inlinecode{no-emoji} nor \inlinecode{emoji} arguments were specified, the metrics are calculated for both
\end{description}
\textbf{Usage and sample output}


\begin{Verbatim}[formatcom=\color{darkgray},fontfamily=courier]
$ python evaluate_api_performance.py --metric accuracy 
  ------------------------------------------------------
  Calculations for api: sentiment_api1_emoji with spam
  Real sentiment distribution {
    "positive": "43.51%", 
    "neutral": "45.61%", 
    "negative": "10.88%"
  }
  True positives: 
    {'positive': 1205, 'neutral': 2167, 'negative': 177} 
  False negatives: 
    {'positive': 1439, 'neutral': 605, 'negative': 484} 
  False positives: 
    {'positive': 517, 'neutral': 1694, 'negative': 317} 
  Total sentiment predictions: 6077 
  Accuracy: 0.584000 
  - - - - - - - - - - - - - - - - - - - - - - - - - - - 
  Updating table im_sentiment_api_stats
  Setting `accuracy_with_spam` = 0.584 
  Where `api_id` = 'sentiment_api1_emoji' 
  ... 1 row affected
  ------------------------------------------------------
\end{Verbatim}


% -- REAL SENTIMENT --
\subsubsection*{Script for determining real sentiment}
\noindent The \emph{\wrapunderscore{update\_real\_sentiment\_and\_spam.py}} script can be invoked via the command line and stores the results in \inlinecode{im\_commento\_sentiment} table.
For each comment the script displays its content and prompts user for input to determine: 
 \begin{enumerate}
  \item whether or not the comment is only a user tag (e.g @Anna)
  \item whether or not the comment is spam and if so, to specify a type
  \item comment sentiment
\end{enumerate}
The script accepts the following optional arguments (if no arguments are specified script runs for all comments):
\begin{description}[labelindent=0.7cm, leftmargin=1.7cm]
\singlespacing
 \item[--help -h ] shows all available arguments and exits
 \item[-idgt \textless nb\textgreater] runs for all comments that satisfy $id > nb$
 \item[-idlt \textless nb\textgreater] runs for all comments that satisfy $id < nb$
 \item[-ideq \textless nb nb\textgreater ] runs for the the space separated list of comment ids
\end{description}
\textbf{Usage and sample output}

\begin{Verbatim}[formatcom=\color{darkgray},fontfamily=courier]
 $ python update_real_sentiment_and_spam.py -ideq 6 7
   --------------------------------------------------
   Comment_id: 6
   Content: Bellissime!
   English translation: Beautiful!
   - - - - - - - - - - - - - - - - - - - - - - - - - 
   Is this comment ONLY a mention? (y/n): n
   Updating table im_commento_sentiment
   Setting `is_mention` = 0
   Where idcommento = 6 
   ... 1 row affected
   - - - - - - - - - - - - - - - - - - - - - - - - - 
   Is this comment spam? (y/n): n
   Updating table im_commento_sentiment
   Setting `spam` = '{"type": "", "is_spam": false}' 
   Where idcommento = 6 
   ... 1 row affected
   - - - - - - - - - - - - - - - - - - - - - - - - - 
   Determine the real_sentiment: pos/neg/neu/mix? pos
   Updating table im_commento_sentiment
   Setting `real_sentiment` = '{
    "sentiment_label": "positive",
     "sentiment_stats": {
       "positive": 1, 
       "negative": 0,
       "neutral": 0
    }}'
   Where idcommento = 6 
   ... 1 row affected
   --------------------------------------------------
\end{Verbatim}

% -- AUTOMATED SENTIMENT ANALYSIS --

\subsubsection*{Automated sentiment analysis}
The sentiment prediction workflow in Figure \ref{fig:prediction-workflow} is implemented by \emph{\wrapunderscore{automated\_sentiment\_analysis.py}} script. 
Each part of the workflow is implemented by an independent script that exports its functionality via s single function. 
Additionally, all of them are also standalone scripts and can be run via the command line.
The following table contains a breakdown of scripts used by \emph{\wrapunderscore{automated\_sentiment\_analysis.py}}.
\begin{table}[H]
\centering
\doublespacing
\begin{tabularx}{\textwidth}{ m{2.5cm} | X }
	\hline
 	translate\_comments.py & The script:
 			\begin{enumerate}
			\singlespacing
			  \item gets comment's original content from \inlinecode{im\_commento}
			  \item calls Google Translate API with comment's content 
		\end{enumerate} 
\end{tabularx}
\end{table}

\newpage
\thispagestyle{empty}
\begin{table}[H]
\centering
\onehalfspacing
\begin{tabularx}{\textwidth}{ m{3cm} | X }
	\hline
 	&   \begin{enumerate}
			\singlespacing
 			\setcounter{enumi}{2}
			  \item stores the translation and detected language in \inlinecode{im\_commento\_sentiment} data table
		\end{enumerate} 
		\textbf{Command line arguments:} -h, -ideq, -idlt, -idlt \\ 
	\hline 
		predict\_comment\_sentiment.py & Depending on the command line arguments or function parameters provided, the script:
		\begin{enumerate}
			\singlespacing
			  \item gets comment's original content from \inlinecode{im\_commento}
			  \item gets comment's English translation content from \inlinecode{im\_commento\_sentiment}
			  \item calls one (or all) APIs for sentiment analysis
			  \item stores the new sentiment prediction in \inlinecode{im\_commento\_sentiment\_emoji} table
		\end{enumerate}
		\textbf{Command line arguments:} -h, -ideq, -idlt, -idlt, -api, --original-language \\ 
	\hline
		update\_sentiment\_prediction\_with\_emojis.py & If comment contains emojis, the script: 
		\begin{enumerate}
			\singlespacing
			  \item gets sentiment stats from \inlinecode{im\_commento\_sentiment} table
			  \item adds the sentiment score from \inlinecode{im\_emoji\_stats} for each emoji
			  \item normalizes stats so the likelihoods of positive, negative and neutral labels add up to 1
			  \item stores the new sentiment prediction in \inlinecode{im\_commento\_sentiment\_emoji} table
		\end{enumerate}
		\textbf{Command line arguments:} -h, -ideq, -idlt, -idlt, -api \\ 
	\hline
		update\_post\_sentiment\_stats.py & Depending on the command line arguments or function parameters provided, the script:
		\begin{enumerate}
			\singlespacing
			  \item gets sentiment stats from \inlinecode{im\_commento\_sentiment} or  
			  \inlinecode{im\_commento\_sentiment\_emoji} table for all post's comments
			  \item count per API how many positive, negative and neutral comments the post has
			  \item normalizes stats so the likelihoods of positive, negative and neutral labels add up to 1
			  \item stores the results of this aggregation in \inlinecode{im\_post\_sentiment\_stats} table
		\end{enumerate}
		\textbf{Command line arguments:} -h, -ideq, -idlt, -idlt, \newline -api\_column \\ 
	\hline
\end{tabularx}
\end{table}
\newpage

\noindent\textbf{Usage and sample output}
\begin{Verbatim}[formatcom=\color{darkgray},fontfamily=courier]
$ python predict_comment_sentiment.py -api ViveknAPI -ideq 6
  --------------------------------------------------
  Comment_id: 6
  Content: Bellissime!
  Translation: Beautiful!
  Updating table im_commento_sentiment
  Setting `sentiment_api1_en` = '{
    "sentiment_stats": {
      "sentiment_label": "positive",
      "positive": 0.724, 
      "negative": 0,
      "neutral": 0,
      }
  }' 
  Where idcommento = 6 
  ... 1 row affected
  --------------------------------------------------
\end{Verbatim}


