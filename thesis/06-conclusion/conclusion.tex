In this final chapter we will summarize all the reasons for involving in such project, as well as the main contributions of the same. After listing the results we will address possible future improvements.

Nowadays with emerging markets and information flow it has become a necessity to try to predict future trends. Thus, companies are processing information luring through Internet with hope they will make a right choice. Big role in company's marketing strategy are social media channels, such as Facebook, Twitter or Instagram. Recognizing the potential use of customers input on the Web, companies have started gathering data related to their online advertisements. Logically, next step was to find a proper way of processing the data in order to discover certain correlations that could guide their production planning. One of recent methods for doing so is called sentiment analysis.

Our report consists of describing current trends in the field of sentiment analysis and how it is applied in business. Showing the reasons for using such method has brought us to idea of investigating about available open source solutions. We have tried to make a comparison with some of sentiment analysis APIs on a given dataset which consists of Facebook comments related to a certain post about fashion industry products. The project itself consists of building a framework representing in a user-friendly way data that has been provided to us with obtained sentiment results of different APIs. We have made a comparison of each API on the data as it is and on data translated to English language. Our statistical results have shown that APIs in general perform better when doing analysis of comments translated in English. Another point where we have seen a potential improvement in our analysis was taking into consideration emojis or emoticons. Currently, emojis have been one of the easiest and most used way of communication. People have seen them as a fast, expressive enough version of typed text. Taking this into consideration we have investigated about finding a proper way to use power of emojis to improve our results. Most obvious solution was building a hash table of emoticons and their related English translations (for example :) equals happy). Guiding our research towards idea of finding a such hash table has brought us to Emoji Sentiment Ranking which contains needed sentiment score for most of emoticons present in our dataset. For ones missing the score we have found a similar from the rank table and assigned the same sentiment score.

After obtaining sentiment scores from different APIs on the various versions of the given data we were able to make a comparison between them. Not only between them, we were able to show how a particular API performs on original data, data translated to English language and also on data taking emoji sentiment score into consideration.
