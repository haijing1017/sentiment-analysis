In this final chapter we will summarize all the reasons for involving in such project, as well as the main contributions of the same. After listing the results we will address possible future improvements.

Nowadays with emerging markets and information flow it has become a necessity to try to predict future trends. Thus, companies are processing information luring through Internet with hope they will make a right choice. Big role in company's marketing strategy are social media channels, such as Facebook, Twitter or Instagram. Recognizing the potential use of customers input on the Web, companies have started gathering data related to their online advertisements. Logically, next step was to find a proper way of processing the data in order to discover certain correlations that could guide their production planning. One of recent methods for doing so is called sentiment analysis.

Our report consists of describing current trends in the field of sentiment analysis and how it is applied in business. Showing the reasons for using such method has brought us to idea of investigating about available open source solutions. We have tried to make a comparison with some of the most known sentiment analysis APIs on a given dataset which consists of Facebook comments related to a certain post about fashion industry products. 
