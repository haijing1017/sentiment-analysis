With the rise of social media the way business and customers interact has drastically changed. 
Business are expected to have a web presence and to produce content in order to engage with their user base on a daily base. 
In return, the users don't shy away from leaving feedback, and they leave a lot of it. 

Having that in mind, it is easy to see how having an automated way to tell if the user-generated content was negative or positive would provide a lot of value for a businesses. 
Coincidentally, in recent years a lot of research and advances have been made in the field of sentiment analysis and the efforts yielded a number of tools for predicting sentiment of textual content.
This is why we were interested in examining the landscape of open source APIs that provide that functionality.

In this thesis we have built a framework for assessing the performance of some  open source  APIs for sentiment analysis. 
The APIs were tested against a dataset of social media content generated by real fashion brands and their user base. 
Because of the global nature of the fashion industry the APIs were appraised on how well they perform in predicting  sentiment of data in original language as well as their English translations. 
Finally, the framework improved  the accuracy of procured sentiment predictions by taking into account the sentimental value of emojis and emoticons found in the data.
