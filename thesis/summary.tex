Con l'avvento dei social media, il modo in cui le aziende interagiscono con i propri clienti è drasticamente cambiato. DA un lato, le aziende devono necessariamente avere una presenza sul Web e produrre contenuti su base giornaliera in grado di coinvolgere i propri clienti e attrarne di nuovi. Dall'altro i clienti non hanno timore a rilasciare numerosi commenti sul Web e soprattutto nei social media.  

In questo scenario, è semplice intravedere del valore aggiunto per le aziende nei metodi automatici in grado di individuare se i contenuti generati dagli utenti hanno un'attitudine negativa o positiva. Infatti negli ultimi anni molte attività di ricerca sono state dedicate alla analisi del "sentiment" e hanno portato alla definizione di diversi strumenti automatici per predire il sentiment dei contenuti testuali rilasciati dagli utenti nel Web. Questo scenario ci ha spinto a esaminare con il nostro lavoro di tesi alcune API open source per la sentiment analysis. In particolare, la tesi riguarda la costruzione di un tool per valutare e confrontare API diverse.

Il tool sviluppato ha permesso di applicare le API su un data set di commenti generati da aziende che realmente operano nel settore della moda e da utenti di social media (Facebook, Twitter) che fanno parte della fan base delle aziende. Vista la natura globale dell'industria della moda, le API sono state applicate sia ai post nel loro linguaggio originale, sia alla loro traduzione in Inglese. Infine, tramite il framework definito, abbiamo provato come l'accuratezza dell'analisi del sentiment possa migliorare se si tiene conto del  significato degli emojis e degli emoticons presenti nei contenuti analizzati.
